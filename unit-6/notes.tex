\subsection{Array Search Specification}

An \emph{array~search} acts on a data structure known as an \emph{array}.
The \emph{imprecise} \emph{problem specification} is to find an element in the array.
Two algorithms that implement array~search are linear~search (slower) and binary~search (faster).
The \emph{precise} problem specification is as follows.

Given an array \( a \) of integers, and an integer \( x \), find an integer \( i \) such that
\begin{itemize}
  \item if there is no index \( j \) such that the \( j \)th element of \( a \) is equal to \( x \), then \( i \) is \( -1 \),
  \item otherwise, the \( i \)th element of \( a \) is equal to \( x \).
\end{itemize}

This specification is ambiguous in that it does not specify which index to return should there be more than one index \( i \) such that the \( i \)th element of \( a \) is equal to \( x \).
In this case, any such \( i \) may be returned.

\subsection{Algorithms}

An algorithm implements a problem specification.
The process for developing an algorithm is as follows.
\begin{enumerate}
  \item Determine the purpose and requirements of the algorithm.
  This is the problem specification.
  \item Develop a solution to the problem.
  This is the algorithm.
  \item Check that the algorithm satisfies the specification.
  This can be done through
  \begin{itemize}
    \item testing (imperfect), or
    \item writing a proof or convincing argument.
  \end{itemize}
  \item If necessary, determine the run-time complexity or space complexity of the algorithm.
  If the algorithm is particularly complex, a better solution may be required.
\end{enumerate}

Testing is an imperfect method of checking whether an algorithm satisfies its specification.
It can only find bugs; it cannot prove that there are no bugs at all.
Sometimes, an algorithm is so simple that it is immediately clear that it satisfies its specification.

\subsection{Linear Search Algorithm}

The linear~search algorithm handles the ambiguity of the array~search specification by selecting the first result index \( i \) from left to right.

\begin{lstlisting}[
  float = !htp,
  title = {The linear array search algorithm.},
  frame = single,
  numbers = left,
  showlines = true,
  language = {Java},
  ]
int linearSearch(int[] a, int x) {
    for(int i = 0; i < a.length, i++) {
        if (a[i] == x) {
            return i;
        }
    }

    return -1;
}
\end{lstlisting}

For an array of length \( n \), if \( x \) is not found by the algorithm, it will loop \( n \) times.
This is the worst-case scenario.
Its worst-case run-time complexity is \( \bigo{n} \) --- linear complexity.
This means that the run time is linearly proportional to the length of the array.
On average, the number of loop iterations is
\begin{equation*}
  \frac{1 + 2 + \ldots + n}{n} = \frac{n + 1}{2}
\end{equation*}

When \( n \) is small, and the search is performed infrequently, linear~search is sufficient.
When \( n \) is large, or the search is performed frequently, the linear run-time complexity is problematic and a more efficient algorithm is necessary.

\subsection{Binary Search Algorithm}

The binary~search algorithm assumes that the array is sorted.
This is known as a \emph{precondition} of the algorithm.
The algorithm works by halving (approximately) the search space on each iteration.
It is analogous to searching for a word in a dictionary.

\begin{lstlisting}[
  float = !htp,
  title = {The binary array search algorithm.},
  frame = single,
  numbers = left,
  showlines = true,
  language = {Java},
]
int binarySearch(int[] a, int x) {
    assert(isSorted(a));

    int left = 0
    int right = a.length - 1;

    while (left <= right) {
        int middle = (left + right) / 2;

        if (a[middle] < x) {
            left = middle + 1;
        } else if (a[middle] > x) {
            right = middle - 1;
        } else {
            return middle;
        }
    }

    return -1;
}
\end{lstlisting}

Since the search space is halved on each iteration, the number of iterations in the worst-case scenario (when \( x \) is not found by the algorithm) is \( \log_{2} n \).
Thus, its worst-case run-time complexity is \( \bigo{\log n} \) --- logarithmic complexity.
This means that the run time is logarithmically proportional to the length of the array.
Binary~search is quicker and more efficient than linear~search, but it is limited in that the array must be sorted before the algorithm can be applied.

Since binary~search is more complex than linear~search, preconditions, \emph{invariants} and postconditions are used to prove its correctness.
An invariant is a property of an object that remains unchanged after operations of a certain type are applied to the object.
Assertions can be used to check these conditions when testing the algorithm.
