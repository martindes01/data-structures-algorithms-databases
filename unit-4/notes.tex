\subsection{Background}

A database is a collection of relations (tables).
Relations consist of attributes (columns).
Data consists of tuples (rows).
A tuple has a value associated with each attribute.
A key is a set of one or more attributes that is unique to each tuple in a relation.

Each attribute has a domain (type) and every value has a type.
There is a special value \texttt{NULL} (undefined) that is a member of all domains.

The schema of a database includes the name, attributes and domains of every relation in the database.
An \emph{instance} of a database comprises the contents of its relations at a given point in time.

\emph{Relational algebra} is a formal language that underpins relational database query languages such as SQL\@.
A \emph{query} operates on one or more relations and returns a relation.
The simplest query in relational algebra is the name of a relation.
This simply returns a copy of the relation.

Relational algebra provides operators to
\begin{itemize}
  \item filter relations,
  \item slice relations, and
  \item combine relations.
\end{itemize}

\begin{table}[htp]
  \centering
  \caption*{The attributes of the \emph{students} relation.}
  \begin{tabular}{llll}
    \toprule
    sID & sName & sFirstDegree & sFDMark \\
    \bottomrule
  \end{tabular}
\end{table}

\begin{table}[htp]
  \centering
  \caption*{The attributes of the \emph{modules} relation.}
  \begin{tabular}{lll}
    \toprule
    mID & mName & mLecturer \\
    \bottomrule
  \end{tabular}
\end{table}

\begin{table}[htp]
  \centering
  \caption*{The attributes of the \emph{studentModules} relation.}
  \begin{tabular}{llll}
    \toprule
    sID & mID & caMark & examMark \\
    \bottomrule
  \end{tabular}
\end{table}

\subsection{Select Operation}

The \emph{select} operation \( \sigma_{\phi}\,R \) returns a new relation that comprises all tuples of a relation \( R \) for which a certain propositional formula \( \phi \) holds.
The propositional formula consists of atomic formulae and the logical operators \( \land \) (and), \( \lor \) (or) and \( \lnot \) (negation).

For example,
\begin{equation*}
  \sigma_{\text{sID}\,>\,1\ \land\ \text{sFDMark}\,>\,77}\,\text{students}
\end{equation*}
is equivalent to the SQL~query
\begin{lstlisting}[language={SQL}]
SELECT * FROM students WHERE sID > 1 AND sFDMark > 77;
\end{lstlisting}

\subsection{Project Operator}

The \emph{project} operation \( \pi_{a_1, \ldots, a_n}\,R \) returns a new relation that comprises all tuples of a relation \( R \) but with only a subset \( \left\{ a_1, \ldots, a_n \right\} \) of its attributes.

For example,
\begin{equation*}
  \pi_{\text{sID},\,\text{examMark}}\,\text{studentModules}
\end{equation*}
is equivalent to the SQL~query
\begin{lstlisting}[language={SQL}]
SELECT sID, examMark FROM studentModules;
\end{lstlisting}

\subsection{Chaining Operations}

Since these relational operations each act on a relation and return a relation, they can be chained.
For example,
\begin{equation*}
  \pi_{\text{sID},\,\text{examMark}}\!\left( \sigma_{\text{examMark}\,>\,70}\,\text{studentModules} \right)
\end{equation*}
is equivalent to the SQL~query
\begin{lstlisting}[language={SQL}]
SELECT sID, examMark FROM (
  SELECT * FROM studentModules WHERE examMark > 70
);
\end{lstlisting}

It is important to note that SQL tables are multisets and, therefore, allow duplicate rows.
Relations in relational algebra, however, are sets, and do not contain duplicate tuples.

\subsection{Cross Product or Cartesian Product}

The \emph{cross~product} or \emph{Cartesian~product} operator \( \times \) is used to combine two relations.
The result is a relation whose attributes are the union of the attributes of the two relations, and whose tuples are the cross~product of the tuples of the two relations.
\begin{equation*}
  R \times S
\end{equation*}

As a notational requirement, if the same attribute name exists in both relations, each corresponding attribute in the result is prefixed by the name of its source relation.

For example,
\begin{equation*}
  \text{students} \times \text{studentModules}
\end{equation*}
is equivalent to the SQL~query
\begin{lstlisting}[language={SQL}]
SELECT student.sID, sName, sFirstDegree, sFDMark,
       studentModules.sID, mID, caMark, examMark
FROM students, studentModules;
\end{lstlisting}

Typically, the cross~product is not useful unless it is filtered using a select operation.
For example,
\begin{equation*}
  \sigma_{\text{students.sID}\,=\,\text{studentModules.sID}}\!\left( \text{students} \times \text{studentModules} \right)
\end{equation*}
is equivalent to the SQL~query
\begin{lstlisting}[language={SQL}]
SELECT student.sID, sName, sFirstDegree, sFDMark,
       studentModules.sID mID, caMark, examMark
FROM students, studentModules
WHERE students.sID = studentModules.sID;
\end{lstlisting}

\subsection{Theta Join}

The \emph{Theta~join} operator \( \bowtie_{\Theta} \) performs the cross~product of two relations and selects all tuples of the result that satisfy the predicate \( \Theta \).
\begin{equation*}
  R \bowtie_{\Theta} S \equiv \sigma_{\Theta}\!\left( R \times S \right)
\end{equation*}
\begin{lstlisting}[language={SQL}]
SELECT * FROM R JOIN S ON <Theta>;
\end{lstlisting}

\subsection{Natural Join}

The \emph{natural~join} operator \( \bowtie \) performs the cross~product of two relations and selects all tuples of the result where each pair of attributes with the same name have the same value.
It then discards one attribute from each duplicate pair.
\begin{equation*}
  R \bowtie S \cong \sigma_{R.\itol{A}\,=\,S.\itol{A}}\!\left( R \times S \right)
\end{equation*}

For example,
\begin{equation*}
  \text{students} \bowtie \text{studentModules} \cong \sigma_{\text{students.sID}\,=\,\text{studentModules.sID}}\!\left( \text{students} \times \text{studentModules} \right)
\end{equation*}
is equivalent to the SQL~query
\begin{lstlisting}[language={SQL}]
SELECT * FROM students NATURAL JOIN studentModules;
\end{lstlisting}

The natural~join does not add expressive power to relational algebra since it can be expressed as a combination of projection, selection and cross~product.

\subsection{Rename Operator}

The \emph{rename} operation \( \rho_{S\left( b_1,\,\ldots,\,b_n \right)} R \) is used to rename a relation \( R \) and its attributes \( \left\{ a_1, \ldots, a_n \right\} \) to a relation \( S \) with attributes \( \left\{ b_1, \ldots, b_n \right\} \).

For example,
\begin{equation*}
  \rho_{S\left( b_1,\,\ldots,\,b_n \right)} R
\end{equation*}
is equivalent to the SQL~query
\begin{lstlisting}[language={SQL}]
SELECT a1 AS b1, ..., an AS bn FROM R AS S;
\end{lstlisting}

A relation can can be renamed without renaming its attributes using the following shorthand notation.
\begin{equation*}
  \rho_{S}\,R
\end{equation*}

As long as a relation has more than one attribute, all of its attributes can be renamed without renaming the relation using the following shorthand notation.
\begin{equation*}
  \rho_{b_1,\,\ldots,\,b_n}\,R
\end{equation*}

The rename operator adds expressive power to relational algebra as it is required in order to refer to specific attributes of a relation formed by a self~join.

\subsection{Self Join}

A \emph{self~join} is any type of join (cross~product, Theta~join or natural~join) whose two operands are the same relation.
In order to refer to specific attributes of a relation formed by a self~join, it is necessary to create renamed copies of the original relation before joining.

For example, a relation that describes pairs of modules taught by the same lecturer can be formed as follows.
The select operation discards tuples in which both modules are the same (\( \text{M1.mName} = \text{M2.mName} \)) or are reversed (\( \text{M1.mName} > \text{M2.mName} \)).
\begin{equation*}
  \begin{split}
    & \pi_{\text{M1.mName},\,\text{M2.mName}}\!\left(\right. \\
    & \quad \sigma_{\text{M1.mName}\,<\,\text{M2.mName}}\!\left(\right. \\
    & \quad \quad \rho_{\text{M1}\left( \text{M1.mID},\,\text{M1.mName},\,\text{mLecturer} \right)} \text{modules} \\
    & \quad \quad \quad \bowtie \\
    & \quad \quad \rho_{\text{M2}\left( \text{M2.mID},\,\text{M2.mName},\,\text{mLecturer} \right)} \text{modules} \\
    & \quad \left.\right) \\
    & \left.\right)
  \end{split}
\end{equation*}

\subsection{Set Operators}

The basic set operators are \emph{union} \( \cup \), \emph{difference} \( - \) and \emph{intersection} \( \cap \).
In relational algebra, these operators may only be applied to relations of the same schema.
In order to find the union of student names (`sName') and module names (`mName'), for example, it is necessary first to project only these attributes from their respective relations, and to give the attributes the same name.

The intersection operator does not add expressive power to relation algebra, as an intersection can be expressed as a combination of differences.
\begin{equation*}
  A \cap B \equiv A - \left( A - B \right)
\end{equation*}

Assuming two relations are of the same schema, their intersection is equivalent to their natural~join.
\begin{equation*}
  A \cap B \equiv A \bowtie B
\end{equation*}
