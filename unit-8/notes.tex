\subsection{AVL Tree Concepts}

\subsubsection{Height of a Node}

The height of a node is the length of the longest path from the node to a leaf node.

\subsubsection{Balance of a Node}

The balance of a node is the difference between the height of its left subtree and the height of its right subtree.

\begin{lstlisting}[mathescape=true]
balance($ \square $) = 0
balance(Fork(x, left, right)) = height(left) - height(right)
\end{lstlisting}

\subsubsection{Perfectly Balanced Binary Tree}

A binary tree is perfectly balanced if the left and right subtrees of any node have the same height.
In other words, the balance of each node is zero.
The height \( h \) and size \( n \) of a perfectly balanced tree are related by the following equations.
\begin{gather*}
  n = 2^{h} - 1 \\
  h = \log_{2} \left( n + 1 \right)
\end{gather*}

The number of nodes grows exponentially as a function of the height.
The height grows logarithmically as a function of the number of nodes.
Even for a small height, there is a large number of nodes.

\subsubsection{AVL Balance}

After many insertions or deletions, a binary search tree is likely to grow imbalanced.
\emph{AVL~trees} -- named after Adelson-Velski and Landis -- solve this problem by assuming additional conditions in order to keep the trees balanced.
An AVL~tree is a binary search tree in which the balance at each node is \( -1 \), \( 0 \) or \( 1 \).
It is difficult to maintain perfectly balanced trees.
AVL~balance is simpler to maintain and is sufficient for fast algorithms.

The size \( n \) of an AVL tree of height \( h \) has an upper bound of the size of a perfectly balanced tree of the same height, and a lower bound of the size of a \emph{Fibonacci~tree} of the same height.
\begin{equation*}
  F\!\left( h + 2 \right) - 1 \leq n \leq 2^{h} - 1
\end{equation*}

The \emph{Fibonacci~sequence} is defined as follows.
\begin{gather*}
  F\!\left( 0 \right) = 0 \\
  F\!\left( 1 \right) = 1 \\
  F\!\left( k + 2 \right) = F\!\left( k \right) + F\!\left( k + 1 \right)
\end{gather*}

Fibonacci numbers can also be calculated using \emph{Binet's formula}.
\begin{equation*}
  F\!\left( k \right) = \frac{\left( \frac{\sqrt{5} + 1}{2} \right)^{k} - \left( \frac{\sqrt{5} - 1}{2} \right)^{k}}{\sqrt{5}} = \bigo{{1.61\ldots}^{k}}
\end{equation*}

Thus, the size \( n \) of an AVL tree of height \( h \) lies in the range
\begin{equation*}
  \bigo{{1.61\ldots}^{h}} \leq n \leq \bigo{2^{h}}
\end{equation*}

This implies that the height \( h \) of an AVL tree of size \( n \) is of the order of the logarithm of \( n \).
\begin{equation*}
  h = \bigo{\log n}
\end{equation*}

\subsection{Algorithms}

The invariant for AVL tree operations is that the balance of each node is \( -1 \), \( 0 \) or \( 1 \).
If this invariant is broken within a loop, it is fixed by rebalancing the tree before the end of the iteration.

The search algorithm for an AVL tree is the same as that for a binary search tree.
Insertion and deletion are the same as those for a binary search tree with the addition of a rebalancing algorithm.

\begin{table}[!htp]
  \centering
  \caption*{Time complexities of algorithms on binary search trees and AVL trees.}
  \begin{tabular}{lcccc}
    \toprule
    Algorithm & \multicolumn{2}{c}{Binary search tree} & \multicolumn{2}{c}{AVL tree} \\
    \cmidrule(l{0.25em}r{0.25em}){2-3} \cmidrule(l{0.25em}r{0.25em}){4-5}
    & Average & Worst case & Average & Worst case \\
    \midrule
    Search & \( \bigo{\log n} \) & \( \bigo{n} \) & \( \bigo{\log n} \) & \( \bigo{\log n} \) \\
    Insert & \( \bigo{\log n} \) & \( \bigo{n} \) & \( \bigo{\log n} \) & \( \bigo{\log n} \) \\
    Delete & \( \bigo{\log n} \) & \( \bigo{n} \) & \( \bigo{\log n} \) & \( \bigo{\log n} \) \\
    \bottomrule
  \end{tabular}
\end{table}

The space complexity is \( \bigo{n} \) for both binary search trees and AVL trees, since both must maintain all of their nodes.

\subsection{Rebalancing an AVL Tree}

If the insertion or deletion of a single element from an AVL tree causes it to lose balance, the tree could exist in one of four possible states of imbalance.
These states can be classified as follows, where \( x \) is the deepest node where an imbalance occurs following an insertion into, or deletion from, subtree~\( z \).

\begin{figure}[!htp]
  \centering
  \begin{tabular}{ccc}
    \begin{tabular}{c}
      \begin{forest}
        [x [y [z [A] [B]] [C]] [D]]
      \end{forest}
    \end{tabular} & \begin{tabular}{c}
      \( \longrightarrow \) \\
    \end{tabular} & \begin{tabular}{c}
      \begin{forest}
        [y [z [A] [B]] [x [C] [D]]]
      \end{forest}
    \end{tabular} \\
  \end{tabular}
  \caption*{A tree in state LL is rebalanced via right rotation at \( x \).}
\end{figure}

\begin{figure}[!htp]
  \centering
  \begin{tabular}{ccc}
    \begin{tabular}{c}
      \begin{forest}
        [x [A] [y [B] [z [C] [D]]]]
      \end{forest}
    \end{tabular} & \begin{tabular}{c}
      \( \longrightarrow \) \\
    \end{tabular} & \begin{tabular}{c}
      \begin{forest}
        [y [x [A] [B]] [z [C] [D]]]
      \end{forest}
    \end{tabular} \\
  \end{tabular}
  \caption*{A tree in state RR is rebalanced via left rotation at \( x \).}
\end{figure}

\begin{figure}[!htp]
  \centering
  \begin{tabular}{ccccc}
    \begin{tabular}{c}
      \begin{forest}
        [x [y [A] [z [B] [C]]] [D]]
      \end{forest}
    \end{tabular} & \begin{tabular}{c}
      \( \longrightarrow \) \\
    \end{tabular} & \begin{tabular}{c}
      \begin{forest}
        [x [z [y [A] [B]] [C]] [D]]
      \end{forest}
    \end{tabular} & \begin{tabular}{c}
      \( \longrightarrow \) \\
    \end{tabular} & \begin{tabular}{c}
      \begin{forest}
        [z [y [A] [B]] [x[C] [D]]]
      \end{forest}
    \end{tabular} \\
  \end{tabular}
  \caption*{A tree in state LR is rebalanced via a left rotation at \( y \) and a right rotation at \( x \).}
\end{figure}

\begin{figure}[!htp]
  \centering
  \begin{tabular}{ccccc}
    \begin{tabular}{c}
      \begin{forest}
        [x [A] [y [z [B] [C]] [D]]]
      \end{forest}
    \end{tabular} & \begin{tabular}{c}
      \( \longrightarrow \) \\
    \end{tabular} & \begin{tabular}{c}
      \begin{forest}
        [x [A] [z [B] [y [C] [D]]]]
      \end{forest}
    \end{tabular} & \begin{tabular}{c}
      \( \longrightarrow \) \\
    \end{tabular} & \begin{tabular}{c}
      \begin{forest}
        [z [x [A] [B]] [y [C] [D]]]
      \end{forest}
    \end{tabular} \\
  \end{tabular}
  \caption*{A tree in state RL is rebalanced via a right rotation at \( y \) and a left rotation at \( x \).}
\end{figure}

Rebalancing is completed by continuing to climb the tree, rebalancing subtrees if necessary, until the root is reached.
