\subsection{Graph Representations}

\subsubsection{Graphs}

A \emph{graph} comprises a finite set of vertices and edges between them.
A graph may be \emph{weighted} or \emph{unweighted}.
Each edge of a weighted graph has a value known as a \emph{weight} associated to it.
A graph may also be \emph{directed} or \emph{undirected}.
Each edge of a directed graph may only be traversed in one direction.
An edge between two vertices \( u \) and~\( v \) in an undirected graph can be interpreted as a pair of directed edges from \( u \) to~\( v \), and from \( v \) to~\( u \).

\subsubsection{Adjacency Matrix}

A graph of \( n \) vertices can be represented by an \( n \times n \) \emph{adjacency~matrix} \( \mathbf{G} \).
For an unweighted graph, each element \( G_{u,v} \) is one if there is an edge from vertex~\( u \) to~\( v \), or zero otherwise.
For a weighted graph, each element \( G_{u,v} \) is the weight of the edge from \( u \) to~\( v \), positive infinity if there is no such edge, or zero if \( u \) and~\( v \) are the same vertex.
A graph is undirected if \( G_{u,v} = G_{v,u} \) for all vertices \( u \) and~\( v \).

\subsubsection{Adjacency Lists}

A graph of \( n \) vertices can also be represented by an array \( \boldsymbol{N} \) of one \emph{adjacency~list} for each vertex (\( n \) lists in total), where each adjacency list \( N_{u} \) is a \emph{linked~list} of the neighbours of vertex~\( u \).
For a weighted graph, each element of the adjacency list \( N_{u} \) is a tuple of a neighbour \( v \) of \( u \) and the weight of the edge from \( u \) to~\( v \).
A vertex~\( v \) is a neighbour of \( u \) if and only if there exists an edge from \( u \) to~\( v \).

\subsubsection{Comparison}

For a graph of \( n \) vertices and~\( m \) edges, an adjacency matrix comprises \( n \) arrays of size \( n \), and has, therefore, a space complexity \( \bigo{n^{2}} \).
The equivalent adjacency list representation comprises \( n \) linked lists that store \( m \) edges in total, and has, therefore, a space complexity of \( \bigo{n + m} \).

Checking for the existence of an edge from \( u \) to~\( v \) in an adjacency matrix involves simply reading the value of \( G_{u,v} \) in \( \bigo{1} \)~time.
Checking for the existence of the edge using adjacency lists involves checking whether the vertex~\( v \) exists in the list \( N_{u} \).
This has linear time complexity in the length of the list \( N_{u} \), which in the worst case is \( \bigo{m} \).

Traversing all neighbours of \( u \) in an adjacency matrix involves reading \( G_{u,v} \) for every vertex~\( v \) in \( \bigo{n} \)~time.
Traversing all neighbours of \( u \) using adjacency lists involves reading only the vertices of the list \( \boldsymbol{N}_{u} \), which in the worst case is \( \bigo{m} \).
Using adjacency lists, only actual neighbours of \( u \) are read.
This is best when the graph is \emph{sparse} (having relatively few edges) rather than \emph{dense}.

\subsection{Shortest Paths}

\subsubsection{Paths}

A \emph{path} from \( u \) to~\( z \) is the sequence of edges between some vertices \( \left( u, v, \ldots, z \right) \) that connect \( u \) to~\( z \).
A path is a \emph{shortest~path} if the sum of the weights of its edges is minimal.

\subsubsection{Dijkstra's Algorithm}

\emph{Dijkstra's algorithm} is an algorithm used to find a shortest path between an origin vertex \( u \) and a destination vertex \( z \) in a graph.
For each vertex~\( v \) in the graph, the algorithm keeps track of the shortest distance~\( d_{v} \) from \( u \) to~\( v \) found so far (initially zero for \( u \), positive infinity otherwise), the predecessor vertex \( p_{v} \) of \( v \) on the path from \( u \) (initially \( v \) itself by convention), and the completion status \( f_{v} \) of the computation of \( d_{v} \) (initially true for \( u \), false otherwise).

The algorithm only works for graphs whose edges have non-negative weights.
There exist alternative algorithms that can handle negative weights but are not as efficient.

The algorithm proceeds as follows.
\begin{enumerate}
  \item Set the current vertex~\( v \) to the origin \( u \).
  \item Set the completion status \( f_{u} \) of \( u \) to true (mark \( u \) as finished).
  \item Repeating until the destination \( z \) is marked as finished,
  \begin{enumerate}
    \item for each neighbour \( w \) of \( v \) such that the sum of \( d_{v} \) and the weight of the edge from \( v \) to~\( w \) is less than \( d_{w} \), set the distance~\( d_{w} \) to that sum and the predecessor~\( p_{w} \) to~\( v \), and
    \item set the current vertex \( v \) to an unfinished vertex (if available) with the smallest distance~\( d_{v} \).
  \end{enumerate}
\end{enumerate}

Initially, \( d_{v} \) is infinity, \( p_{v} = v \) and \( f_{v} \) is false for all vertices \( v \).
The arrays \( \boldsymbol{d} \) and~\( \boldsymbol{f} \) obey the following invariants.
\begin{itemize}
  \item The distance~\( d_{v} \) is the length of the shortest path from \( u \) to~\( v \) passing through only the finished vertices (those for which \( f_{v} \) is true).
  \item If \( v \) is marked as finished (\( f_{v} \) is true), the distance~\( d_{v} \) is the actual length of the shortest path from \( u \) to~\( v \).
\end{itemize}
The weight of the edge from \( u \) to~\( v \) is obtained from the adjacency matrix or adjacency lists that represent the graph.
When the algorithm is finished, the shortest path from \( u \) to~\( z \) is obtained in reverse order from the predecessor array, starting from \( p_{z} \).
Thus, the shortest path is the sequence of edges between the vertices \( \left( u, \ldots, p_{p_{z}}, p_{z}, z \right) \).

The critical steps of the algorithm are to find the next unfinished vertex with the smallest distance from the origin, and to update every neighbour of the current vertex.
Using an adjacency matrix, both of these steps can be completed in \( \bigo{n} \)~time.
Since the steps are repeated up~to \( n \)~times, the overall time complexity of the algorithm when representing the graph using an adjacency matrix is \( \bigo{n^{2}} \).

When the graph is represented using adjacency lists, the algorithm updates all the neighbours of a different vertex on each iteration.
Over all the iterations, a total of \( m \) vertices are updated.
Thus, the update steps contribute a time complexity of \( \bigo{m} \).
The search and update steps can be made more efficient using a \emph{minimum~priority queue} in which the priority of a vertex~\( v \) is the distance~\( d_{v} \).
Using a \emph{binary~heap} implementation for the binary queue, the overall time complexity of the algorithm using adjacency lists is \( \bigo{\left( n + m \right) \log n} \).
Using a \emph{Fibonacci~heap} implementation, the time complexity is \( \bigo{\left( n \log n \right) + m} \).

If the graph is dense (the number of edges \( m \) is approximately \( n^{2} \)), using adjacency lists with a binary heap has time complexity \( \bigo{n^{2} \log n} \), which is slower than using an adjacency matrix.
Using adjacency lists with a Fibonacci heap, however, has a better time complexity of \( \bigo{n^{2}} \) for a dense graph.
If the graph is sparse, using adjacency lists with either type of heap is faster than using an adjacency matrix.

\subsection{Minimal Spanning Trees}

\subsubsection{Spanning Trees}

Considering only undirected and connected graphs, a \emph{spanning~tree} is a minimal selection of edges that connects all vertices.
The selection of edges is minimal if it does not form a cycle in the graph.
A \emph{minimum spanning~tree} is a spanning tree for which the sum of the weights of its edges is minimal.

\subsubsection{Jarn\'{i}k-Prim Algorithm}

The \emph{Jarn\'{i}k-Prim algorithm} forms a minimum spanning tree of a graph by iteratively extending the tree with an edge that connects a vertex that is not yet part of the tree and has a minimal weight.
For each vertex~\( v \) in the graph, the algorithm keeps track of the current distance~\( d_{v} \) of \( v \) from the tree (initially zero for the initial vertex, positive infinity otherwise), the predecessor vertex \( p_{v} \) by which \( v \) is directly connected to the tree (initially \( v \) itself by convention), and the connection status \( f_{v} \) that indicates whether \( v \) has been added to the tree (initially true for the initial vertex, false otherwise).

Unlike Dijkstra's algorithm, the Jarn\'{i}k-Prim algorithm works for graphs whose edges have negative weights.

The algorithm proceeds as follows.
\begin{enumerate}
  \item Set the current vertex~\( v \) to any initial vertex \( u \).
  \item Set the distance~\( d_{u} \) of \( u \) to zero.
  \item Repeating while there is still at least one node marked as unconnected,
  \begin{enumerate}
    \item set the connection status \( f_{v} \) of \( v \) to true (mark \( v \) as connected), and
    \item for each neighbour \( w \) of \( v \) such that the weight of the edge from \( v \) to~\( w \) is less than \( d_{w} \), set the distance~\( d_{w} \) to that weight and the predecessor~\( p_{w} \) to~\( v \), and
    \item set the current vertex \( v \) to an unconnected vertex (if available) with the smallest distance~\( d_{v} \).
  \end{enumerate}
\end{enumerate}

The arrays \( \boldsymbol{d} \), \( \boldsymbol{p} \) and~\( \boldsymbol{f} \) obey the following invariants.
\begin{itemize}
  \item All vertices marked as connected are included in the tree.
  \item For a vertex~\( v \) that is not connected, the distance~\( d_{v} \) is the smallest weight of an edge that can connect \( v \) to the tree.
  \item For any vertex~\( v \), the predecessor~\( p_{v} \) is the vertex of the tree such that the edge from \( p_{v} \) to~\( v \) has weight \( d_{v} \).
\end{itemize}
When the algorithm is finished, the minimum spanning tree is obtained from the predecessor array as the collection of edges from \( p_{v} \) to~\( v \) for every vertex~\( v \) other than the initial vertex \( u \).

The time complexity of the Jarn\'{i}k-Prim algorithm is the same as that of Dijkstra's algorithm.
Using an adjacency matrix, it is \( \bigo{n^{2}} \).
Using adjacency lists with a binary heap, it is \( \bigo{\left( n + m \right) \log n)} \).
Using adjacency lists with a Fibonacci heap, it is \( \bigo{\left( n \log n \right) + m} \).
