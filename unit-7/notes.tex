\subsection{Manipulation of a Sorted Array}

To delete an element from a sorted array
\begin{enumerate}
  \item find the element to be deleted --- \( \bigo{\log n} \) time complexity (efficient), and
  \item shift the following elements left --- \( \bigo{n} \) time complexity (inefficient).
\end{enumerate}
Thus, deleting an element from a sorted array has \( \bigo{n} \) time complexity in both the worst case and on average.

To insert an element into a sorted array
\begin{enumerate}
  \item find the position for insertion --- \( \bigo{\log n} \) time complexity (efficient), and, before insertion,
  \item shift the following elements right --- \( \bigo{n} \) time complexity (inefficient).
\end{enumerate}
Thus, inserting an element into a sorted array has \( \bigo{n} \) time complexity in both the worst case and on average.
It also has the added complexity of having to check the capacity of the array.

Although it is efficient to find elements in a sorted array using binary~search, it is not efficient to insert into, or delete from, a sorted array.
Alternative data structures, such as \emph{sorted~trees} should be considered for applications that require frequent insertion and deletion.

\subsection{Binary Tree}

A \emph{binary~tree} is a data structure whose elements, or \emph{nodes} each have two \emph{subtrees}.
A subtree may be an empty tree.
A node whose subtrees are both empty is a \emph{leaf~node}.
Each non-empty node of a binary~tree has a label that represents the value of an element.
An element can be of any data~type.
All elements in a tree have the same type.

The fundamental rules for building a binary~tree are as follows.
\begin{enumerate}
  \item It is possible to have an empty~tree, denoted by \texttt{Empty} or \( \square \).
  \item Given a label \texttt{x} and two trees, \texttt{left} and \texttt{right}, a new tree \texttt{Fork(x, left, right)} can be built with \texttt{left} and \texttt{right} as the left and right subtrees, respectively, of a \emph{root} node with label \texttt{x}.
  % example diagram
\end{enumerate}
Thus, all binary~trees can be built using only the empty~tree \( \square \) and the fork tree \texttt{Fork}.

The \emph{size} of a tree is the number of non-empty nodes it contains.
The \emph{height} of a tree is the maximum number of edges needed to reach an empty subtree from the root.
Formulae for the size and height of a binary~tree can be defined as follows.
\begin{lstlisting}[mathescape=true]
#nodes($ \square $) = 0
#nodes(Fork(x, left, right))
    = 1 + #nodes(left) + #nodes(right)

height($ \square $) = 0
height(Fork(x, left, right))
    = 1 + max(height(left), height(right))
\end{lstlisting}

\subsection{Perfectly Balanced Binary Tree}

A binary~tree is \emph{perfectly balanced} if the left and right subtrees of any node have the same height.
\begin{lstlisting}[mathescape=true]
isPB($ \square $) = true
isPB(Fork(x, left, right))
    = (height(left) == height(right))
      && isPB(left)
      && isPB(right)
\end{lstlisting}

The number of nodes at level~\( n \) of a perfectly balanced binary~tree, where \( n \) increases from zero at the root, is \( 2^{n} \).
Thus, the total number of non-empty nodes in a perfectly balanced binary~tree of height \( H \) is given by
\begin{equation*}
  \sum_{n=0}^{H-1} 2^{n} = 2^{H} - 1
\end{equation*}

Formulae relating the size and height of a perfectly balanced binary~tree can be defined as follows.
\begin{lstlisting}[mathescape=true]
#nodes(t) = 2^height(t) - 1
height(t) = log(#nodes(t) + 1)
\end{lstlisting}

The height of a balanced binary~tree is significantly smaller than its size.
Thus, a large balanced, or nearly balanced, binary~tree has a small height.
This property can be exploited in order to organise data in such a way that elements can be found, inserted and deleted quickly.
This relies on the tree being sorted.

\subsection{Binary Search Tree (BST)}

\subsubsection{Sorted Tree}

A tree is sorted if
\begin{itemize}
  \item the value of each node in the left subtree is less than the value of the root,
  \item the value of each node in the right subtree is greater than the value of the root, and
  \item the left and right subtrees are themselves sorted.
\end{itemize}
It follows that a sorted tree has no repetitions, although some implementations do allow repetitions.

A \emph{binary~search tree} (BST) is simply a sorted binary~tree.
It is not necessarily balanced.
Formulae to assert that a binary~tree is sorted can be defined as follows.
\begin{lstlisting}[mathescape=true]
isBST($ \square $) = true
isBST(Fork(x, left, right))
    = smaller(left, x) && bigger(right, x)
      && isBST(left) && isBST(right)

smaller($ \square $, x) = true
smaller(Fork(y, left, right), x)
    = (y < x) && smaller(left, x) && smaller(right, x)

bigger($ \square $, x) = true
bigger(Fork(y, left, right), x)
    = (y > x) && bigger(left, x) && bigger(right, x)
\end{lstlisting}

\subsubsection{Search}

The search specification for a BST is as follows.
Given an BST \( t \), and a search value \( x \), find an integer \( i \) such that
\begin{itemize}
  \item if there is no index \( j \) such that the \( j \)th element of \( t \) is equal to \( x \), then \( i \) is \( -1 \),
  \item otherwise, the \( i \)th element of \( t \) is equal to \( x \).
\end{itemize}

The corresponding search algorithm is as follows.
\begin{enumerate}
  \item Starting at the root node, and
  \item repeating until an empty tree is reached,
  \begin{itemize}
    \item if the value of the current node is less than the search value, move to the root node of the left subtree,
    \item otherwise, if the value of the current node is greater than the search value, move to the root node of the right subtree,
    \item otherwise, the value of the current node must be equal to the search value; the search value has been found.
    Return the current position.
  \end{itemize}
  \item Return \( -1 \).
\end{enumerate}

The efficiency of this algorithm is determined by the number of comparisons it performs.
After each comparison, the search moves to the next subtree.
Thus, the maximum number of comparisons is the height of the tree.
If the tree is balanced, or nearly balanced, both the average and worst-case time complexities are \( \bigo{\log n} \).
If the tree becomes highly imbalanced, the worst-case time complexity increases to \( \bigo{n} \).

Formulae to check whether an element \texttt{x} is in a BST \texttt{t} can be defined as follows.
\begin{lstlisting}[mathescape=true]
isIn(x, t) = $ \begin{cases}
  \texttt{true} & \text{// if x occurs in t} \\
  \texttt{false} & \text{// otherwise}
\end{cases} $

isIn(x, $ \square $) = false
isIn(x, Fork(y, left, right))
    = (x == y)
      || ((x < y) && isIn(x, left))
      || ((x > y) && isIn(x, right))
\end{lstlisting}

\subsubsection{Insertion}

The insertion algorithm proceeds in a similar fashion to the search algorithm.
Thus, it has the same time complexities.
Instead of returning a position, the new element is inserted in the correct position.
The action to perform if the insertion element already exists in the tree is determined by the precise problem specification.
The resulting tree is a sorted binary tree (a BST).
After many items are inserted into a BST, it is likely to be imbalanced.
To maintain the efficiency of the BST, efficient balancing algorithms must be considered.

Assuming that the tree is to be left unchanged if it already contains the element to be inserted, insertion formulae can be defined as follows.
\begin{lstlisting}[mathescape=true]
insert(x, $ \square $) = Fork(x, $ \square $, $ \square $)
insert(x, Fork(y, left, right))
    = $ \begin{cases}
      \text{Fork(x, left, right)} & \text{// if x == y} \\
      \text{Fork(y, insert(x, left), right)} & \text{// if x < y} \\
      \text{Fork(y, left, insert(x, right))} & \text{// if x > y}
    \end{cases} $
\end{lstlisting}

\subsubsection{Deletion}

There are several cases to consider for the deletion algorithm.
\begin{itemize}
  \item If the element to delete is a leaf node (with two empty subtrees), the node is simply replaced with an empty subtree.
  The node is found in \( \bigo{\log n} \) time, and deleted in one step.
  The overall time complexity is \( \bigo{\log n} \).
  \item If the element to delete is an internal node with only one empty subtree, the node is replaced by the non-empty subtree.
  This also has \( \bigo{\log n} \) time complexity.
  \item If the element to delete is an internal node with no empty subtrees, the node must be replaced in a manner than maintains the order of the tree.
  One option is to replace the target node with the left-most leaf~node of the right subtree (the leaf node is removed from the subtree).
  Alternatively, the target node could be replaced with the right-most leaf~node of the left subtree.
  This also has \( \bigo{\log n} \) time complexity.
\end{itemize}

The deletion algorithm has the same time complexities as the search and insertion algorithms.
After many elements are removed from a BST, it is likely to be imbalanced.

Assuming that the tree is to be left unchanged if it already contains the element to be deleted, deletion formulae can be defined as follows.
\begin{lstlisting}[mathescape=true]
delete(x, $ \square $) = $ \square $

delete(x, Fork(y, left, right))
    = $ \begin{cases}
      \text{Fork(y, delete(x, left), right)} & \text{// if x < y} \\
      \text{Fork(y, left, delete(x, right))} & \text{// if x > y}
    \end{cases} $

delete(x, Fork(x, left, right))
    = Fork(smallest(right), left, delete(smallest(right), right))
    // if both left and right are non-empty

delete(x, Fork(x, $ \square $, right)) = right
delete(x, Fork(x, left, $ \square $)) = left
delete(x, Fork(x, $ \square $, $ \square $)) = $ \square $
\end{lstlisting}

\subsubsection{Summary}

If the BST is balanced, or nearly balanced, the search, insertion and deletion algorithms all have a time complexity of \( \bigo{\log n} \) in both the worst case and on average.
If the BST is highly imbalanced, the algorithms have a worst-case time complexity of \( \bigo{n} \).

After many insertions or deletions, a BST is likely to be imbalanced.
To maintain the efficiency of the BST, efficient balancing algorithms must be considered.
